\documentclass[t]{beamer}
\usecolortheme{beaver} 
\useinnertheme{circles}
\useinnertheme{rectangles}

%%% Работа с русским языком
\usepackage{cmap}					% поиск в PDF
\usepackage{mathtext} 				% русские буквы в формулах
\usepackage[T2A]{fontenc}			% кодировка
\usepackage[utf8]{inputenc}			% кодировка исходного текста
\usepackage[english,russian]{babel}	% локализация и переносы

%%% Дополнительная работа с математикой
\usepackage{amsmath,amsfonts,amssymb,amsthm,mathtools} % AMS
\usepackage{icomma} % "Умная" запятая: $0,2$ --- число, $0, 2$ --- перечисление

%% Свои команды
\DeclareMathOperator{\sgn}{\mathop{sgn}}

%% Перенос знаков в формулах (по Львовскому)
\newcommand*{\hm}[1]{#1\nobreak\discretionary{}
{\hbox{$\mathsurround=0pt #1$}}{}}

%%% Другие пакеты
\usepackage{lastpage} % Узнать, сколько всего страниц в документе.
\usepackage{soul} % Модификаторы начертания
\usepackage{csquotes} % Еще инструменты для ссылок
%\usepackage[style=authoryear,maxcitenames=2,backend=biber,sorting=nty]{biblatex}
\usepackage{multicol} % Несколько колонок
\usepackage{listings}
\usepackage{relsize}
\usepackage{hyperref}


\title{d-phi-enc (HackTM CTF)}

\author{Кулигин Данил и Борзенко Михаил}
\date{1 июля 2023 г.}
\institute[БФУ им. Иммануила Канта]{БФУ им. Иммануила Канта ИФМНиИТ \\ Компьютерная безопасность, 3 курс}

\begin{document}

\frame[plain]{\titlepage}	% Титульный слайд

\section{Условие задачи d-phi-enc}
\subsection{Описание с CryptoHack}
\begin{frame}
	\frametitle{\insertsection}
	\framesubtitle{\insertsubsection}
    \begin{itemize}
        Описание к заданию: \slshape"В CTF есть много людей, которые ошибочно шифруют p, q в RSA. Но на этот раз..."\upshape\newline\newline
        
        Так же к задаче прилагаются файлы \slshape chall.py \upshape и \slshape output.txt.\upshape
        output.txt содержит \texttt{n} - произведение p и q, \texttt{enc\char`_d} - зашифрованная секретная экспонента, \texttt{enc\char`_phi} - зашифрованная функция Эйлера, \texttt{enc\char`_flag} - зашифрованное сообщение. В \slshape chall.py \upshape содержится код для генерации этих значений и открытая экспонента \texttt{e}\newline\newline
    
        Понятно, что нам нужно каким-то образом извлечь \texttt{d} или \texttt{phi} на основе \texttt{enc\char`_d} и \texttt{enc\char`_phi}.
    \end{itemize}
\end{frame}

\section{Анализ задачи d-phi-enc}
\subsection{Предварительный анализ}
\begin{frame}
	\frametitle{\insertsection}
	\framesubtitle{\insertsubsection}
    \begin{itemize}
        Для начала проведем анализ данных нам чисел и проверим криптосистему на уязвимость к изученным нами атакам, в числе которых: Атака повторным шифрованием, Атака Винера, Атака встреча поседение, Атака методом Ферма.\newline\newline
        Нам даны:\newline
        $\alert{\texttt{n}} = 2447638356...$\newline
        $\alert{\texttt{enc\char`_d}} = 2385197103...$\newline
        $\alert{\texttt{enc\char`_phi}} = 3988439673...$\newline
        $\alert{\texttt{enc\char`_flag}} = 2403368891...$\newline
        $\alert{\texttt{e}} = 3$\newline\newline
        Упрощая, нам дано \boldsymbol{\alert{n}} длиной 2048 бит. Также мы знаем, что длина \boldsymbol{\alert{p}} и \boldsymbol{\alert{q}} равна 1024 бита. Очевидно, что разложить \boldsymbol{\alert{n}} на множители вряд ли будет эффективным решением. 
    \end{itemize}
\end{frame}

\section{Анализ задачи d-phi-enc}
\subsection{Предварительный анализ}
\begin{frame}
	\frametitle{\insertsection}
	\framesubtitle{\insertsubsection}
    \begin{itemize}
        \item Атака “встреча посередине” не будет успешной, т. к. обычно она срабатывает при малой длине сообщения ($l < 64$ бит) и наличии разложения на два примерно равных множителя, битовая длина которых меньше l/2.\newline
        \item Атака методом Ферма также не принесёт результатов, т. к. данная атака могла бы быть успешна, если p и q “близкие” друг к другу чилса (половина старших цифр числа равны).\newline
        \item Атака Винера гарантированно успешна, при $d<d_{кр}$, где $\alert{d_{кр}} = n^{1/4}/\sqrt{2a}$, $\alert{a} = (h+1)/\sqrt{h}$, $\alert{h} = p/q$. 
    \end{itemize}
\end{frame}

\section{Анализ задачи d-phi-enc}
\subsection{Предварительный анализ}
\begin{frame}
	\frametitle{\insertsection}
	\framesubtitle{\insertsubsection}
    \begin{itemize}
        \item Как мы можем заметить, для вычисления $d_{кр}$ требуется знать \boldsymbol{\alert{p}} и \boldsymbol{\alert{q}}, а мы их пока что не знаем, но мы знаем точно, что попытка атаки Винера (метод которой заключается в разложении дроби \boldsymbol{\alert{e}}/\boldsymbol{\alert{n}} в цепную дробь для последующего нахождения \boldsymbol{\alert{d}}) не принесла результата.\newline
        \item Атака повторным шифрованием также не дала быстрого результата, из чего можем сделать вывод, что порядок \boldsymbol{\alert{e}} по модулю сообщения достаточно велик.
    \end{itemize}
\end{frame}

\section{Решение задачи d-phi-enc}
\subsection{Теоретические рассчёты}
\begin{frame}[t] % [t], [c], или [b] --- вертикальное выравнивание на слайде (верх, центр, низ)
	\frametitle{\insertsection}
	\framesubtitle{\insertsubsection}
	\begin{itemize}
		\item Из алгоритма работы RSA мы знаем, что d это обратное число к e по          модулю $\varphi(n)$.\newline
            \begin{center}
                \( e*d1 \equiv \pmod{\varphi(n)} \)\newline
                \( e*d = k1*\varphi(n) +1 \)\newline
            \end{center}
            \item Т. к. $\boldsymbol{\alert{d}} < \boldsymbol{\alert{n}}$ и \boldsymbol{\alert{e}} = 3 можем сделать вывод, что $0 < k1 < 3$. Далее распишем $d_{enc}$ используя сравнение выше.\newline
            \begin{center}
                \( d_{enc} \equiv d^3 \pmod{n} \)\newline
                \( e^3*d_{enc} \equiv e^3*d^3 \pmod{n} \)\newline
                \( 27d_{enc} \equiv (e*d)^3 \pmod{n} \)\newline
                \( 27d_{enc} \equiv (k1*\varphi(n) +1) \pmod{n} \)\newline
                \( 27d_{enc} \equiv (k1^3*\varphi^3(n) + 3k1^2*\varphi^2(n) + 3k1*\varphi(n) + 1) \pmod{n} \)\newline
		\end{center}
	\end{itemize}
\end{frame}

\section{Решение задачи d-phi-enc}
\subsection{Теоретические рассчёты}
\begin{frame}
	\frametitle{\insertsection}
	\framesubtitle{\insertsubsection}
	\begin{itemize}
            \item Заменим $\varphi^3(n)$ на $\varphi_{enc}$, т. к.\boldsymbol{\alert{e}} = 3, а сравнение выполняется по модулю \boldsymbol{\alert{n}}, также как и шифрование.\newline\newline
            \( 27*d_{enc} \equiv (k1^3*\varphi_{enc} + 3k1^2*\varphi^2(n) + 3k1*\varphi(n) + 1) \pmod{n} \)\newline
            \item Перенеся всё в одну сторону можно получить квадратное сравнение от $\varphi(n)$\newline\newline
            \( 3k1^2*2(n)+3k1*(n)+k1^3*\varphi_{enc} - 27d_{enc} + 1 \equiv 0 \pmod{n} \)\newline
	\end{itemize}
\end{frame}

\section{Решение задачи d-phi-enc}
\subsection{Теоретические рассчёты}
\begin{frame}[t] % [t], [c], или [b] --- вертикальное выравнивание на слайде (верх, центр, низ)
	\frametitle{\insertsection}
	\framesubtitle{\insertsubsection}
	\begin{itemize}
            \item Все переменные кроме $\varphi(n)$ даны. Однако, т.к. \boldsymbol{\alert{n}} не является простым числом, то решить такое сравнение можно только через систему сравнений по модулям множителей \boldsymbol{\alert{n}}, но в нашем случае множители неизвестны. Поэтому попробуем расписать $\varphi(n)$ для упрощения решения.\newline\newline
            \begin{center}
            \( \varphi(n)=(p-1)*(q-1)=pq-p-q+1=n-p-q+1 \)\newline
            \( \varphi(n)-(p+q)+1 \pmod{n} \)
            \end{center}
	\end{itemize}
\end{frame}
\section{Решение задачи d-phi-enc}
\subsection{Теоретические рассчёты}
\begin{frame}[t] % [t], [c], или [b] --- вертикальное выравнивание на слайде (верх, центр, низ)
	\frametitle{\insertsection}
	\framesubtitle{\insertsubsection}
        \begin{itemize}
            \item Обозначим $x=p+q$, тогда\newline
            \begin{center}
                \( 3k1^2*\varphi^2(n) + 3k1*\varphi(n) + k1^3*\varphi_{enc} - 27d_{enc} + 1 \equiv 0 \pmod{n} \)\newline
                \( 3k1^2*(1-x)^2 + 3k1*(1-x) + k1^3*\varphi_{enc} - 27d_{enc} + 1 \equiv 0 \pmod{n} \)\newline
                \( 3k1^2*(1-2*x+x^2)^2 + 3k1*(1-x) + k1^3*\varphi_{enc} - 27d_{enc} + 1 \equiv 0 \pmod{n} \)\newline
                \( 3k1^2*x^2 + (-6k1^2-3k1)*x + (3k1^2+3k1+k1^3*\varphi_{enc} - 27d_{enc} + 1) \equiv 0 \pmod{n} (1) \)
            \end{center}
	\end{itemize}
\end{frame}

\section{Решение задачи d-phi-enc}
\subsection{Теоретические рассчёты}
\begin{frame}[t] % [t], [c], или [b] --- вертикальное выравнивание на слайде (верх, центр, низ)
	\frametitle{\insertsection}
	\framesubtitle{\insertsubsection}
        \begin{itemize}
            \item Далее зная что $x = p + q$, можем сделать вывод что x гораздо меньше n. Допустим p>q.
            \begin{center}
                \( x^2=(p+q)2=p^2+q^2+2*p*q=p^2+q^2+2n \)\newline
            \end{center}
            
            \item Зная что \boldsymbol{\alert{p}}, \boldsymbol{\alert{q}} сгенерированы длиной 1024 бит, $p/q < 2$. Можем записать такое неравенство:\newline
            $p\leq 2*q$, следовательно $p^2 \geq 4*q^2$. А также $q^2 \leq n$\newline
            $p^2+q^2+2*n \leq 4*q^2+q^2+2*n$\newline
            $x^2 \leq 5*q^2+2*n$\newline
            $x^2 \leq 5*n+2*n$\newline
            $x^2 \leq 7*n$\newline

	\end{itemize}
\end{frame}

\section{Решение задачи d-phi-enc}
\subsection{Теоретические рассчёты}
\begin{frame}[t] % [t], [c], или [b] --- вертикальное выравнивание на слайде (верх, центр, низ)
	\frametitle{\insertsection}
	\framesubtitle{\insertsubsection}
        \begin{itemize}
            \item Используя коэффициент a из квадратного сравнения (1) распишем неравенство для $x^2$.\newline
            $3k1^2*x^2 \leq 3*2^2*x^2$, так как $k1 \leq 2$.\newline
            $3k1^2*x^2 \leq 12*7*n$\newline
            $3k1^2*x^2 \leq 84*n$\newline
            $3k1^2*x^2+(-6k1^2-3k1)*x+(3k1^2+3k1+k1^3*\varphi_{enc}-27d_{enc}+1) < 84n$\newline
            \item Мы знаем все числа из \boldsymbol{b} и \boldsymbol{c} коэффициентов неравенства выше, делаем вывод что \boldsymbol{b} и \boldsymbol{c} отрицательны, поэтому меняем знак на строгий.\newline
            $3k1^2*x^2+(-6k1^2-3k1)*x+(3k1^2+3k1+k1^3*\varphi_{enc}-27d_{enc}+1)=k2*n$,
            где $k2 \leq 84$, $k2 \in Z$.
	\end{itemize}
\end{frame}

\section{Решение задачи d-phi-enc}
\subsection{Теоретические рассчёты}
\begin{frame}[t] % [t], [c], или [b] --- вертикальное выравнивание на слайде (верх, центр, низ)
	\frametitle{\insertsection}
	\framesubtitle{\insertsubsection}
        \begin{itemize}
            \item Таким образом мы можем перебрать все возможные \boldsymbol{k2} чтобы решить уравнение, корнем которого будет \boldsymbol{x}, $x=p+q$.
            \item Для решения таска нам нужно знать $\varphi(n)$, оно находится с помощью выражения ниже, но мы также можем подсчитать p и q чтобы проверить наше решение $p*q=n$.\newline
            $\varphi(n)=n-p-q+1$, домножим на p и перенесем всё в одну сторону.\newline\newline
            $pn-p^2-pq+p-p*\varphi(n)=0$\newline
            $p^2-pn-p+p*\varphi(n)+n=0$\newline
            $p^2-p*(n-\varphi(n)+1)+n=0$
	\end{itemize}
\end{frame}

\section{Решение задачи d-phi-enc}
\subsection{Теоретические рассчёты}
\begin{frame}[t] % [t], [c], или [b] --- вертикальное выравнивание на слайде (верх, центр, низ)
	\frametitle{\insertsection}
	\framesubtitle{\insertsubsection}
        \begin{itemize}
            \item Решением данного квадратного уравнения будут p и q. Далее мы можем проверить выражение $p*q=n$ и переходить к финальному шагу.\newline
            Вычисляем d = $e^{-1}\pmod{\varphi(n)}$\newline
            Вычисляем флаг: flag = \texttt{$enc\char`_flag^d \pmod{n}$}\newline
            \item С помощью функции \texttt{long\char`_to\char`_bytes} вычисляем текстовое значение флага, которое равно:\newline
            \texttt{b"HackTM$\{$Have you warmed up? If not, I suggest you consider the case where e=65537, although I don't know if it's solvable. Why did I say that? Because I have to make this flag much longer to avoid solving it just by calculating the cubic root of enc$\char`_$flag.$\}$"}
	\end{itemize}
\end{frame}

\section{Решение задачи d-phi-enc}
\subsection{Код для решения данной задачи}
\begin{frame}[fragile]
	\frametitle{\insertsection} 
	\framesubtitle{\insertsubsection}
	\footnotesize
	\smaller
\begin{lstlisting}[language=Python]
from Crypto.Util.number import long_to_bytes

n = 2447638356...
enc_d = 2385197103...
enc_phi = 3988439673...
enc_flag = 2403368891
e = 3

for k1 in range(1, 3):
    a = 3*(k1^2)
    b = -(6*(k1^2)+3*k1)
    c = 3*(k1^2) + 3*k1 + (k1^3)*enc_phi - 27*int(enc_d) + 1

    #f = a*(x^2) + b*x + c
    det = b^2 - 4*a*c
    for k2 in range(85):
        c -= n
        det = b^2 - 4*a*c
        if(is_square(det)):break
    if(is_square(det)):break
\end{lstlisting}	
\end{frame}

\section{Решение задачи d-phi-enc}
\subsection{Код для решения данной задачи}
\begin{frame}[fragile]
	\frametitle{\insertsection} 
	\framesubtitle{\insertsubsection}
	\footnotesize
	\smaller
\begin{lstlisting}[language=Python]
qrs_det = sqrt(b^2 - 4*a*c)
print("qrs_det=", qrs_det)

cand_x = (-1*b + qrs_det)/(2*a) #x=p+q
phi = n - cand_x + 1
print("phi=", phi)

p = ((n + 1 - phi) + sqrt(((n + 1 - phi) ^ 2) - 4 * n))/ 2
q = n // p
print("p: ", type(p), p)
print("q: ", type(q), q)
assert(p*q == n)

d = pow(e, -1, phi)
print("flag=", (pow(enc_flag, d, n)))
\end{lstlisting}	
\end{frame}
\section{Концепция RSA}
\subsection{Математическая модель}

\begin{frame}[t] % [t], [c], или [b] --- вертикальное выравнивание на слайде (верх, центр, низ)
	\frametitle{\insertsection}
	\framesubtitle{\insertsubsection}
	\begin{itemize}
		\item<1-> Асимметричные криптографические системы основаны на так называемых односторонних функциях с секретом.\newline 
		Например, операция возведения числа в степень по модулю: \newline 
		\begin{center} \( c\equiv f(m) \equiv m^e \pmod{n} \) \newline
		\( m\equiv f^{-1}(c) \equiv c^d \pmod{n} \) \newline 
		\( d\equiv e^{-1} \pmod{\varphi(n)} \) \newline \end{center}
		\item<2-> \boldsymbol{\alert{c}} получено возведением в степень по модулю числа m. Назовём это действие шифрованием
		\newline 
		\boldsymbol{\alert{m}} - открытый текст, а \boldsymbol{\alert{c}} - шифртекст
		\newline 
		Пара чисел (\boldsymbol{\alert{e}}, \boldsymbol{\alert{n}}) - открытый ключ
		\newline
		\boldsymbol{\alert{d}} - закрытый ключ 
	\end{itemize}
\end{frame}

\subsection{Математическая модель}

\begin{frame}
	\frametitle{\insertsection}
	\framesubtitle{\insertsubsection}
    \begin{itemize}
        \item<1-> 
            Пусть $ \boldsymbol{\alert{n}} = p \cdot q $, где где p и q – некоторые разные простые числа. Для такого n функция Эйлера имеет вид: \newline 
            \begin{center} $\varphi(n) = (p - 1) \cdot (q - 1)$ \end{center}
            \vspace{3mm}
        \item<2-> 
            Выберем такое число \boldsymbol{\alert{e}}: \newline 
            \begin{center}
               \hspace{10mm} $ e \in [3, \varphi(n) - 1]$ \newline 
                $НОД(e, \varphi(n) - 1) = 1$
            \end{center}
            Вычислим \alert{\boldsymbol{d}}:  \newline 
            \begin{center}
                \( d\equiv e^{-1} \pmod{\varphi(n)} \)
            \end{center}
    \end{itemize}
\end{frame}

\begin{frame}
    \frametitle{\insertsection}
	\framesubtitle{\insertsubsection}
	\begin{itemize}
	    \item<1-> Возьмём в качестве сообщения число $m \in [1, n-1]$ \newline 
	    Чтобы зашифровать его, необходимо возвести \newline его в степень \boldsymbol{\alert{e}} по модулю \boldsymbol{\alert{n}} \newline 
	    \begin{center}
	         \( c\equiv m^{e} \pmod{n} \)
	         \vspace{3mm}
	    \end{center}
	    Отметим также, что $\boldsymbol{\alert{c}} \in [1, n-1]$, как и \boldsymbol{\alert{m}}. Расшифруем шифртекст, возведя его в степень закрытого ключа \boldsymbol{\alert{d}}:
	    
	    \begin{center}
	    \vspace{3mm}
	        \( m` \equiv c^d \pmod{n} \)
	        \vspace{3mm}
	    \end{center}
	\end{itemize}
\end{frame}


\section{Используемая литература}

\begin{frame}
    \frametitle{\insertsection}
    \begin{thebibliography}{1}
    \bibitem{2}Математика криптографии и теория шифрования. Лекция 13: Квадратичное сравнение.\newline
    https://intuit.ru/studies/courses/552/408/lecture/9370
    \bibitem{2}RSA: от простых чисел до электронной подписи \newline 
    https://habr.com/ru/post/534014/
    \bibitem{2}Криптоанализ RSA. Сонг Ян, 2011 год. ISBN 978-5-93972-873-7.
    \bibitem{2}The CrypTool Book: Learning and Experiencing Cryptography with CrypTool and SageMath. Prof. Bernhard Esslinger and the Development Team of the Open-Source Software CrypTool. Edition 12 (2018).\newline
    https://www.cryptool.org
    \end{thebibliography}
\end{frame}

\end{document}